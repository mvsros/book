\documentclass{article}
\usepackage{graphicx}
\usepackage{subcaption}
\usepackage[T2A]{fontenc}
\usepackage[utf8x]{inputenc}
\usepackage[english, ukrainian]{babel}
\usepackage{url}
\usepackage{latexsym,amsmath,amsfonts,amssymb,amsthm,mathrsfs}
\usepackage{cite,hhline}
\usepackage{hyperref}
\usepackage{graphicx,mwe}
\usepackage{wrapfig}
\usepackage{array}
\usepackage{tabularx, multirow}
\usepackage{longtable}
\usepackage{lipsum} % just for dummy text- not needed for a longtable

\addto\captionsukrainian{\renewcommand{\figurename}{Мал.}}
\usepackage{book}					% стильовий файл журналу 

\title{Мій перший документ}
\date{2022-11-14}
\author{МАН }



\begin{document}
	\maketitle
	\newpage
	
\Title{Оформлення шаблону книги}\label{radap1xxx:FirstPage}
\Authors{Мосійчук В.С.}
\aff{Київський політехнічний інститут імені Ігоря Сікорського, м. Київ, Україна}
\Address{}{\href{mailto:mvs@ros.kpi.ua}{mvs@ros.kpi.ua}}
\AbsKeywords{У роботі представлені вимоги щодо оформлення статей для подання у збірник наукових праць ``Вісник Національного технічного університету України <<Київський політехнічний інститут>>. Серія Радіотехніка. Радіоапаратобудування''. Показано, що дотримання встановлених правил дозволить покращити вашу статтю .
}{Ключові слова}{правила оформлення, радіотехніка, радіоапаратобудування, \LaTeX}
	


	
\section{Тест підтримки української мови}
	
Obviously the statements title, date and author are not within the the document environment, so they will not directly show up in our document. The area before our main document is called preamble. In this specific example we use it to set up the values for the maketitle command for later use in our document. This command will automagically create a titlepage for us. The newpage command speaks for itself.

If we now compile again, we will see a nicely formatted title page, but we can spot a page number at the bottom of our title page. What if we decide, that actually, we don’t want to have that page number showing up there. We can remove it, by telling LaTeX to hide the page number for our first page. This can be done by adding the pagenumbering{gobble} command and then changing it back to \pagenumbering{arabic} on the next page numbers like so:

%Привіт світ

Формула кола: $ y^2 + x^2 = 1  $
$$
 y = \sqrt{x+1}
$$

\[
y = \sqrt{x^2+1}
\]


\begin{equation}\label{eq1}
	H(A)=-\sum \limits_{i=1}^n p_i \log p(p_i)
\end{equation}

\begin{equation}\label{eq2}
	H(A)=-\sum \limits_{i=1}^n p_i 
\end{equation}

That’s it. You’ve successfully created your first LaTeX document. The following lessons will cover how to structure your document and we will then proceed to make use of many features of LaTeX.

\subsection{Begin subsection}
Obviously the statements title, date and author are not within the the document environment, so they will not directly show up in our document. The area before our main document is called preamble. In this specific example we use it to set up the values for the maketitle command for later use in our document. This command will automagically create a titlepage for us. The newpage command speaks for itself.

\begin{wrapfigure}{L}{0.5\textwidth}
%\begin{figure}[htp]
	\raggedleft %note: there's going to be another image next to this later
	\includegraphics[width=5cm]{example-image-a}\\
	\noindent
	- Point 1 under the first description\\
	\caption{\label{fig:frog1} Підпис до рисунку.}
\end{wrapfigure} 
If we now compile again, we will see a nicely formatted title page, but we can spot a page number at the bottom of our title page. What if we decide, that actually, we don’t want to have that page number showing up there. We can remove it, by telling LaTeX to hide the page number for our first page. This can be done by adding the pagenumbering{gobble} command and then changing it back to \pagenumbering{arabic} on the next page numbers like so:

\begin{wrapfigure}{R}{0.4\textwidth}
	\raggedright
	\includegraphics[width=0.38\textwidth]{example-image-b}\\
	\caption{\label{fig:frog2} Підпис до рисунку.}
\end{wrapfigure}
If we now compile again, we will see a nicely formatted title page (Мал. \ref{fig:frog1}), but we can spot a page number at the bottom of our title page. What if we decide, that actually, we don’t want to have that page number showing up there. We can remove it, by telling LaTeX to hide the page number for our first page. This can be done by adding the pagenumbering{gobble} command and then changing it back to \pagenumbering{arabic} on the next page numbers like so:
If we now compile again, we will see a nicely formatted title page, but we can spot a page number at the bottom of our title page. What if we decide, that actually, we don’t want to have that page number showing up there. We can remove it, by telling LaTeX to hide the page number for our first page. This can be done by adding the pagenumbering{gobble} command and then changing it back to \pagenumbering{arabic} on the next page numbers like so:


If we now compile again, we will see a nicely formatted title page, but we can spot a page number at the bottom of our title page. What if we decide, that actually, we don’t want to have that page number showing up there. We can remove it, by telling LaTeX to hide the page number for our first page. This can be done by adding the pagenumbering{gobble} command and then changing it back to \pagenumbering{arabic} on the next page numbers like so:


	
\section{Таблиці}
	
Obviously the statements title, date and author are not within the the document environment, so they will not directly show up in our document. The area before our main document is called preamble. In this specific example we use it to set up the values for the maketitle command for later use in our document. This command will automagically create a titlepage for us. The newpage command speaks for itself.

Obviously the statements title, date and author are not within the the document environment, so they will not directly show up in our document. The area before our main document is called preamble. In this specific example we use it to set up the values for the maketitle command for later use in our document. This command will automagically create a titlepage for us. The newpage command speaks for itself.

Obviously the statements title, date and author are not within the the document environment, so they will not directly show up in our document. The area before our main document is called preamble. In this specific example we use it to set up the values for the maketitle command for later use in our document. This command will automagically create a titlepage for us. The newpage command speaks for itself.

Obviously the statements title, date and author are not within the the document environment, so they will not directly show up in our document. The area before our main document is called preamble. In this specific example we use it to set up the values for the maketitle command for later use in our document. This command will automagically create a titlepage for us. The newpage command speaks for itself.

\begin{center}
	\begin{tabular}{| l | c | r |}
		\hline
		dummy text dummy text  & cell2 & cell3 \\ 
		\hline
		cell4 & cell5 & cell6 \\ 
		\hline 
		cell7 & cell8 & dummy text  \\   
		\hline
	\end{tabular}
\end{center}

\begin{center}
	\begin{tabular}{||c c c c||} 
		\hline
		Col1 & Col2 & Col2 & Col3 \\ 
		\hline
		1 & 6 & 87837 & 787 \\ 
		\hline
		2 & 7 & 78 & 5415 \\
		\hline
		3 & 545 & 778 & 7507 \\
		\hline
		4 & 545 & 18744 & 7560 \\
		\hline
		5 & 88 & 788 & 6344 \\ 
		\hline
	\end{tabular}
\end{center}


\begin{center}
	\begin{tabular}{ | m{6em} | m{2cm}| m{4cm} | } 
		\hline
		cell1 dummy text dummy text dummy text& cell2 & cell3 \\ 
		\hline
		cell1 dummy text dummy text dummy text & cell5 & cell6 \\ 
		\hline
		cell7 & cell8 & cell9 \\ 
		\hline
	\end{tabular}
\end{center}

\begin{tabularx}{0.98\textwidth} { 
		| >{\raggedright\arraybackslash}X 
		| >{\centering\arraybackslash}c 
		| >{\raggedleft\arraybackslash}c | }
	\hline
	cell1 dummy text dummy text dummy text cell1 dummy text dummy text dummy text & item 12 & item 13 \\
	\hline
	item 21  & item 22  & item 23  \\
	\hline
\end{tabularx}

\subsection{Об'єднання комірок}

\begin{tabular}{ |p{3cm}||p{3cm}|p{3cm}|p{3cm}|  }
	\hline
	\multicolumn{4}{|c|}{Country List} \\
	\hline
	Country Name or Area Name& ISO ALPHA 2 Code &ISO ALPHA 3 Code&ISO numeric Code\\
	\hline
	Afghanistan   & AF    &AFG&   004\\
	Aland Islands &   AX  & ALA   &248\\
	Albania 	  &AL & ALB&  008\\
	Algeria       &DZ & DZA&  012\\
	American Samoa&   AS  & ASM&016\\
	Andorra& AD  & AND   &020\\
	Angola& AO  & AGO&024\\
	\hline
\end{tabular}


\begin{center}
	\begin{tabular}{ |c|c|c|c| } 
		\hline
		col1 & col2 & col3 \\
		\hline
		\multirow{3}{4em}{Multiple row} & cell2 & cell3 \\ 
		& cell5 & cell6 \\ 
		& cell8 & cell9 \\ 
		\hline
	\end{tabular}
\end{center}

\section{Довгі таблиці}

\lipsum[1]
\lipsum[1]


%\begin{table}[h] 
%\centering
\newpage
\begin{longtable}{| p{.18\textwidth} | p{.7\textwidth} |} 
	\hline
	foo & bar \\ \hline 
	foo & bar \\ \hline
	foo & bar \\ \hline
	foo & bar \\ \hline
	foo & bar \\ \hline
	foo & bar \\ \hline
	foo & bar \\ \hline
	foo & bar \\ \hline
	foo & bar \\ \hline
	foo & bar \\ \hline
	foo & bar \\ \hline
		foo & bar \\ \hline
	foo & bar \\ \hline
	foo & bar \\ \hline
	foo & bar \\ \hline
	foo & bar \\ \hline
	foo & bar \\ \hline
	foo & bar \\ \hline
		foo & bar \\ \hline
	foo & bar \\ \hline
	foo & bar \\ \hline
	foo & bar \\ \hline
	foo & bar \\ \hline
	foo & bar \\ \hline
	foo & bar \\ \hline
		foo & bar \\ \hline
	foo & bar \\ \hline
	foo & bar \\ \hline
	foo & bar \\ \hline
	foo & bar \\ \hline
	foo & bar \\ \hline
	foo & bar \\ \hline
		foo & bar \\ \hline
	foo & bar \\ \hline
	foo & bar \\ \hline
	foo & bar \\ \hline
	foo & bar \\ \hline
	foo & bar \\ \hline
	foo & bar \\ \hline
		foo & bar \\ \hline
	foo & bar \\ \hline
	foo & bar \\ \hline
	foo & bar \\ \hline
	foo & bar \\ \hline
	foo & bar \\ \hline
	foo & bar \\ \hline
		foo & bar \\ \hline
	foo & bar \\ \hline
	foo & bar \\ \hline
	foo & bar \\ \hline
	foo & bar \\ \hline
	foo & bar \\ \hline
	foo & bar \\ \hline
		foo & bar \\ \hline
	foo & bar \\ \hline
	foo & bar \\ \hline
	foo & bar \\ \hline
	foo & bar \\ \hline
	foo & bar \\ \hline
	foo & bar \\ \hline
		foo & bar \\ \hline
	foo & bar \\ \hline
	foo & bar \\ \hline
	foo & bar \\ \hline
	foo & bar \\ \hline
	foo & bar \\ \hline

	\caption{Your caption here} % needs to go inside longtable environment
	\label{tab:myfirstlongtable}
\end{longtable}
%\end{table} 

Table \ref{tab:myfirstlongtable} shows my first longtable.

\lipsum[1]



\section*{Висновки}

A document has a preamble and document part
The document environment must be defined
\begin{itemize}
	\item Commands beginning with a backslash \, environments have a begin and end tag
	\item Useful settings for pagenumbering:
	\item gobble – no numbers
	\item arabic – arabic numbers
	\item roman – roman numbers
	\item roman – roman numbers2
	\item roman – roman numbers3
	\item roman – roman numbers4
	\item roman – roman numbers5
\end{itemize}


Копище
Засноване у 1459 році.

Населення – 983.

Площа – 2,83 км².

Поштовий індекс – 11010.

Середня висота над рівнем моря – 156 м.


Історія села Копище

Копище — село розташоване на березі річки Уборті (притока Прип’яті), за 55 км від районного центру та залізничної станції Олевськ.

Після Андрусівського перемир’я 1667 року село залишилося у складі Польщі. Боротьбу проти своїх поневолювачів селяни не припиняли. Вони активно боролися під час визвольного повстанського руху проти польсько-шляхетських загарбників’ за возз’єднання Правобережної України з Росією під проводом С. Палія у кінці XVII — на початку XVIII століття.

Страшних бідувань зазнали селяни Копища в раки першої світової війни. Відчувалася нестача робочих рук, тягла, скоротилися посівні площі, голод і хвороби косили людей.

Звістку про Лютневу буржуазно-демократичну революцію 1917 року в село принесли солдати, які поверталися з фронту. Всі чекали змін у житті, вимагали розподілу земель, але Тимчасовий уряд не поспішав поліпшувати становище трудового народу.

Активними борцями за нове життя були копищанські жінки. Так, у постійно діючих комісіях при сільраді їх працювало 4, у комнезамі — 3, у касі взаємодопомоги — 3, у шкільній раді — 3, у правлінні кредитного товариства — 3. На з’їзд жінок Олевського району 1925 року від Конища поїхало 16 жінок.

1927 року в селі Копищі було організовано перше колективне господарство — ТСОЗ, що охопив спершу 12 господарств, а на початку 1928 року — вже 35. Товариство мало всього лише 8 коней, 4 плуги, 4 борони, 2 культиватори та сівалку. В перші місяці його існування держава надала кредити в розмірі 7690 крб. на будівництво господарчих приміщень, придбання робочої худоби, на закладення саду.

1936 року в селі діяли медична амбулаторія, пологовий будинок на 8 ліжок, де, крім лікаря, працювали дві акушерки та дві санітарки. Під час жнив відкривалися дитячі ясла на 120 місць.

1936 року збудовано нове приміщення семирічної школи, де 8 учителів навчали 300 учнів. Добра слава йшла про роботу Копищанського клубу. Тут читалися лекції й доповіді. У залі на 200 місць ставилися концерти силами гуртківців, двічі на місяць демонструвалися кінокартини. Сільська бібліотека мала книжковий фонд 6 тис. примірників. Товарами народного споживання трудящих забезпечували два магазини.

13 липня 1941 року в село (вдерлися фашисти. Настали чорні дні неволі. Гітлерівці грабували населення, забирали хліб, худобу. 70 мирних жителів вивезли на каторжні роботи до Німеччини.

13 липня 1943 року в с.Копище сталася одна з найжахливіших трагедій України: нацистські загарбники влаштували каральну операцію “Фрау Хельга” (пані Ольга), метою якої було повне знищення жителів села. За один день спалили живими 2887 мирних жителів, з них 1347 дітей віком до 12 років. Врятувались лише четверо дорослих та 46 дітей.

Незважаючи на величезні труднощі, копищанці всі сили віддавали на відродження рідного села. Стало легше працювати, коли після перемоги над гітлерівською Німеччиною до села почали повертатися демобілізовані чоловіки.

Поступово піднімалося з руїн Копище. Вже через рік після перемоги тут виросла ціла вулиця. Один з будинків пристосували під лікарняний пункт, інший під початкову школу.

Основні сільськогосподарські культури у колгоспі — льон, картопля, зернові. Важливу роль у його економіці відіграє тваринництво м’ясо-молочного напряму. На 1 січня 1967 року колгосп мав 685 голів великої рогатої худоби, в т. ч. 275 корів та 199 голів овець. Для їх розміщення споруджено під шифером 5 добротних тваринницьких приміщень. На той час господарство було повністю електрифіковано, мало 9 тракторів, 6 комбайнів, 8 автомашин. Діяла пилорама.

На тваринницьких фермах обладнано автоматичне напування.

Копищанці справжні господарі своєї долі. 35 чоловік обрано депутатами сільської Ради, серед яких 15 жінок, 11 чоловік — членами правління колгоспу. В постійно діючих комісіях працює близько 100 чоловік. Бюджет Ради на 1973 рік становить 39 964 крб., переважна більшість цієї суми буде використана на поліпшення охорони здоров’я, на піднесення культурно-освітньої роботи.

В центрі Копища знаходиться Братьска могила жертвам фатишизму (1961 р.), в якій поховано 2887 жителів села, утому числі 1347 дітей.

Поруч знаходиться музей, який розповідає людям про історію рідного села і про страшну трагедію, яку не можна забути. Директор музею – Мартиновець Марія Василівна.

У селі наразі є загальноосвітня школа І-ІІІ ступенів. Директор – Лозко Катерина Сафронівна. Дітей навчається 186. Також функціонує заклад дошкільної освіти з короткотривалим перебуванням дітей без харчування, який відвідує 22 дітей.

Відзначення 75-ї річниці Копищенської трагедії

Сучасне освітлення в селі Копище



\section{Вивчаємо фізику за допомогою Scratch}

\Authors{Мізюк Олександр Миколайович}
\aff{Механіка. Вільне падіння і криволінійний рух під дією незмінної сили тяжіння}
\AbsKeywords{У цьому розділі представлений зразок розв'язування задачі на рух тіла кинутого під кутом до горизонту. Для моделювання процесу руху використовуйте застосунок, створений у середовищі програмування \href{https://scratch.mit.edu/projects/editor/?tutorial=getStarted}{Scratch}.
}{Ключові слова}{механіка, криволінійний рух, Scratch, \LaTeX}

\subsection{Рух тіла, кинутого під кутом до горизонту}

Прочитавши про рекорди швидкості польоту спортивних снарядів, учениця Оленка вирішила з’ясувати, якої швидкості вона надає футбольному м’ячу. Для цього дівчинка вдарила по м’ячу, спрямувавши його під кутом \textbf{$45^{\circ}$} до горизонту (див. мал. \ref{fig:hit1}).

\begin{figure}[h]
	\centering
	\includegraphics[width=0.3\linewidth]{./images-miziuk/hit1.png}
	\caption{
		\centering
		За напрямком і дальністю польоту м’яча ви можете визначити, якої швидкості ви надали м’ячу під час удару або кидка.}
	\label{fig:hit1}
\end{figure}

М’яч упав на землю на відстані \textbf{40 м} від учениці. Виконавши розрахунки, дівчинка вирішила, що вона надала м’ячу швидкості \textbf{20 м/с}, а м’яч піднявся на висоту \textbf{8 м}. Чи не помилилася учениця?

\subsection{Розв'язуємо задачу\protect\footnote{Ознайомтеся з розв’язанням задачі. Скориставшись отриманими формулами, оцініть розрахунки дівчинки, а після уроків проведіть подібний експеримент та оцініть швидкість, якої ви надаєте м’ячу.}}

Футболістка вдарила по м’ячу, надавши йому швидкості $v_0$, напрямленої під кутом $\alpha$ до горизонту. Визначте дальність польоту та найбільшу висоту підйому м’яча.

\begin{center}
	\begin{tabular}{|l|l|} 
		\hline
		Дано & \textit{Розв'язання} \\
		\hline
		$v_0$ & \\
		$\alpha$ & \\
		$g$ & \\	
		\hline
		$L - ?$ & \\
		$h_{max} - ?$	& \\
		\hline
	\end{tabular}
\end{center}

Виконаємо пояснювальний рисунок (див. мал. \ref{fig:hit2}): початок координат пов’яжемо з точкою на поверхні Землі, де м’яч відірвався від бутси футболістки; вісь \textbf{OY} спрямуємо вертикально вгору; вісь \textbf{ОХ} — горизонтально.

\begin{figure}[h]
	\centering
	\includegraphics[width=0.6\linewidth]{./images-miziuk/hit2.png}
	\caption{
		\centering
		Пояснювальний рисунок.}
	\label{fig:hit2}
\end{figure}

В обраній системі відліку:

рух уздовж осі \textbf{ОХ} — рівномірний:

\begin{equation}\label{eq:hinteq1}
	v_x = v_{0x}, x = x_0 + v_{0x}t,
\end{equation}

де 

$$
	x_0 = 0, v_{0x} = v_0cos\alpha 
$$

рух уздовж осі \textbf{ОY} — рівноприскорений:

\begin{equation}\label{eq:hinteq2}
	v_y = v_{0y} + g_yt, y = y_0 + v_{0y}t + \dfrac{g_yt^2}{2},
\end{equation}

де 

$$
	y_0 = 0, v_{0y} = v_0sin\alpha, g_y = -g,
$$

тому рівняння \eqref{eq:hinteq1} і \eqref{eq:hinteq2} набувають вигляду:

\begin{equation}\label{eq:hinteq3}
	v_x = v_0cos\alpha, x = v_0cos\alpha\cdot t
\end{equation}

і

\begin{equation}\label{eq:hinteq4}
	v_y = v_0sin\alpha - gt, y = v_0sin\alpha\cdot t - \dfrac{gt^2}{2}
\end{equation}

відповідно. Час \textbf{t1} руху м’яча до верхньої точки траєкторії (точки \textbf{А}) знайдемо з умови: $v_y(t_1) = 0$:

\begin{equation}\label{eq:hinteq5}
	v_0sin\alpha - gt_1 = 0 \Rightarrow t_1 = \dfrac{v_0sin\alpha}{g}.
\end{equation}

Координата y м’яча в точці \textbf{А} — це максимальна висота підйому м’яча:

\begin{equation}\label{eq:hinteq6}
	h_{max} = y_A = v_0sin\alpha\cdot t_1 - \dfrac{gt_1^2}{2}.
\end{equation}

Після підстановки \textbf{t1} отримуємо формули для визначення максимальної висоти підйому та загального часу руху м’яча:

\begin{equation}\label{eq:hinteq7}
	h_{max} = \dfrac{v_0^2sin\alpha^2}{2g};
\end{equation}

\begin{equation}\label{eq:hinteq8}
	t = 2t_1 = \dfrac{2v_0sin\alpha}{g}.
\end{equation}

Дальність \textbf{L} польоту дорівнює координаті \textbf{х} тіла наприкінці руху (\textbf{x = L}):

\begin{equation}\label{eq:hinteq9}
	x = v_0cos\alpha\cdot t = v_0cos\alpha\cdot \dfrac{2v_0sin\alpha}{g}. 
\end{equation}

Отже, дальність польоту:

\begin{equation}\label{eq:hinteq10}
	2cos\alpha\cdot sin\alpha \Rightarrow L = \dfrac{v_0^2sin2\alpha}{g}. 
\end{equation}

\begin{figure}[h]
	\centering
	\includegraphics[width=0.6\linewidth]{./images-miziuk/hit3.png}
	\caption{
		\centering
		Однакова дальність польоту при різних траєкторіях.}
	\label{fig:hit3}
\end{figure}

\textbf{Зверніть увагу!} З останньої формули випливає:

\begin{itemize}
	\item якщо кинути тіло під кутом $\alpha$, а потім під кутом $90^{\circ} - \alpha$, то дальність польоту не зміниться, тобто тіло потрапить у ту саму точку, рухаючись різними траєкторіями (див. мал. \ref{fig:hit3})
	\item максимальної дальності польоту тіло сягає, якщо $\alpha = 45^{\circ}$ ($sin2\alpha = 1$).
\end{itemize}

\subsection{Моделювання руху у Scratch}

\begin{figure}[h]
	\centering
	\includegraphics[width=1\linewidth]{./images-miziuk/hit4.png}
	\caption{
		\centering
		Знімок вікна проєкту.}
	\label{fig:hit4}
\end{figure}

Щоб розглянути код проєкту, \href{https://scratch.mit.edu/projects/681697866}{перейдіть за покликанням}. 

\subsection{Підбиваємо підсумки}

Траєкторія руху тіла, кинутого під кутом до горизонту, - параболічна. Такі рухи розглядають як результат додавання двох простих рухів: горизонтального - рівномірного уздовж осі \textbf{OX} і вертикального - рівноприскореного (з прискоренням \textbf{g}) уздовж осі \textbf{OY}. 

Рівняння залежностей проекції швидкості та координати від часу у цьому разі мають вигляд:

$$
	v_x = v_{0x}, 
$$

$$
	x = x_0 + v_{0x}t,
$$

$$
	v_y = v_{0y} + g_yt, 
$$

$$
	y = y_0 + v_{0y}t + \dfrac{g_yt^2}{2}.
$$

\subsection{Контрольні запитання}

\begin{itemize}
	\item Який вигляд має рівняння руху, якщо тіло кинуто під кутом до горизонту?
	\item Якою є траєкторія руху тіла, кинутого під кутом до горизонту? Наведіть приклади.
	\item Як визначити модуль і напрямок швидкості руху тіла в будь-якій точці траєкторії?
\end{itemize}

\subsection{Вправа для самостійного розв'язування}

Опором повітря нехтуйте. Вважайте, що $g = 10 {}^2$ % 
{м/с}.\\

Струмінь води, напрямлений під кутом \textbf{$60^{\circ}$} до горизонту, сягнув висоти \textbf{15 м}.

\begin{enumerate}
	\item Використайте розроблений Scratch-проєкт для дослідження руху струменя води.
	\item Знайдіть: 
	\begin{enumerate}
		\item швидкість витікання води; 
		\item час польоту частинок струменя; 
		\item дальність польоту частинок струменя.
	\end{enumerate}
	\item Якою буде дальність струменя, якщо спрямувати його під кутом \textbf{$30^{\circ}$} до горизонту?
	\item Чому струмінь води розширюється?	
\end{enumerate}

\end{document}
