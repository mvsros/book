 \documentclass[12pt]{article} % Обрано клас документу та розмір шрифта
\usepackage[ukrainian]{babel} % Підтримка української мови
\usepackage { graphicx }
\title{Обухівка} 
\date{\today}
\author{Старотіторов Валерій }
\begin{document}
	\maketitle	              
	
Обухівка розташована в центральній частині Дніпропетровської області на лівому березі Дніпровського водосховища на Дніпрі. Неподалік Обухівки знаходиться місце впадіння нового русла річки Оріль у Дніпро. Зі сходу до Обухівки впритул прилягають лівобережні околиці міста Дніпра. На східній околиці з озера Чередниченкове витікає річка Шпакова. На півночі межує із селом Горянівське.

Обухівка розташована у фізико-географічній зоні Придніпровська низовина. Висота над рівнем моря у селищі — 50-55 метрів.

Довкола селища розташовані Обухівські озера.

На захід від Обухівки лежать землі Дніпровсько-Орільського природного заповідника.
\begin{figure}
	\centering
	\includegraphics[width=0.7\linewidth]{25}
	\caption{Арт-об'єкт "Я кохаю Обухівку"}
	\label{fig:25}
\end{figure}


Історія
На території Обухівки досліджено кілька поселень доби бронзи, катакомбної та зрубної культур (кінець ІІІ — початок І тисячоліття до н. е.).

Засноване в другій половині XVIII століття. Першою назвою поселення була Обухівка, яка, за неофіційними даними, походить від прізвища першого поселенця — козака Андрія Обуха.

Після утворення Катеринославської губернії Обухівка ввійшла до складу Кам'янської волості Новомосковського повіту. У 1886 році тут мешкало 1930 осіб з 351 господарським подвір'ям, була церква, 3 магазини.

В перші роки радянської влади в Обухівці, як і у всій країні, пройшла колективізація. Станом на 1925 рік в селі мешкало 5 246 жителів, проводилось два ярмарки, було дві школи, хата-читальня[1]. Обухівка входила до складу Кам'янського району Катеринославської округи.

У 1938 році, за рішенням загальних зборів селян Обухівки, вирішено перейменувати село на Кіровське на честь радянського державного діяча Сергія Кірова.[2] Згодом до селища була приєднано придніпровське село Сугаківка. Обухівка увійшла до складу новоутвореного Дніпропетровського району.

Після Другої світової війни Кіровське отримало статус селища міського типу. Тут була розташована центральна садиба Дніпропетровської птахофабрики. У другій половині 20 століття у Кіровському побудовано понад 1 200 нових житлових будинків, Палац культури, дві бібліотеки, лікарська амбулаторія.

Деякий час Обухівка підпорядковувалася Дніпропетровській міській раді, з 1965 року, після відновлення Дніпропетровського району, знову входить до нього.

% TODO: \usepackage{graphicx} required
\begin{figure}[h]
	\centering
	\includegraphics[width=0.8\linewidth]{23}
	\caption{\label{fig:frog3}Храм Святих мучениць Віри, Надії, Любові та матері їх Софії}
	\label{fig:23}
\end{figure}

Сучасність
Більшість (90 %) працюючого населення Обухівки працює у місті Дніпрі.

У селищі є дві середні загальноосвітні школи, дві амбулаторії загальної практики—сімейної медицини, будинок культури.

Дитячий садок, збудований у 2018 році, у лютому 2020 року номінований організаторами конкурсу Mies van der Rohe Award на премію ЄС у галузі сучасної архітектури.[3]
\begin{figure}
	\centering
	\includegraphics[width=0.8\linewidth]{24}
	\caption{	\centering Обухівський дитсадок Дізнатись більше: https://litsa.com.ua/dityachij-sadok-v-obuxivci-nominuvali-na-premiyu-yes/}
	\label{fig:24}
\end{figure}

Обухівка відіграє роль «дачного селища» та місця відпочинку для жителів міста Дніпра, оскільки за часів радянської влади тут було розміщено багато дач та баз відпочинку (в основному для оздоровлення робітників місцевих заводів та фабрик). У 2017 році в Обухівці створено громадську організацію "Дитячий Туристичний Клуб «БАФ», якій об'єднав понад 100 активних громадян селища. У 2018 році в Обухівці створено «Молодіжний центр А. Т.О. М.», центр є майданчиком для зустрічей активної молоді селища

Герб
Орлан і пернач засвідчують давні козацькі традиції, пов’язують зі символікою Орільської полкової паланки, на території громади річка Оріль впадає у Дніпро. Орлан є символом волі й благородства. Три зірки означають сходження тут меж колишніх Орільської, Протовчанської та Кодацької паланок, вказують близькість до міста Дніпра (на гербі якого такі ж зірки) та характеризують об’єднану громаду, до складу якої входять три населені пункти. Хвиляста основа підкреслює розташування поселення біля річки Дніпра. Зелене поле символізує щедру природу, Дніпровсько-Орільський природний заповідник. Дубова та ялинова гілки підкреслюють багатство місцевої флори, є символом міцності та сили.

Транспорт
Через селище проходять автомобільні дороги Т 0404 і Н31.

Обухівка сполучена приміським транспортом (маршрутками) з Дніпром.

Пам'ятки
Храм на честь Воздвиження Чесного і Животворящого Хреста Господнього
Поблизу Обухівки наявний регіональний ландшафтний парк Дніпровські ліси.
Відомі люди
Уродженцями села є

Варфоломєєв Володимир Олександрович (1984—2014) — старший солдат Збройних сил України, учасник російсько-української війни.
Лисенко Олег Ігорович (1989—2016) — молодший сержант Збройних сил України, учасник російсько-української війни.
	
\end{document}