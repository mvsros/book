	
\section{Таблиці}
	
Obviously the statements title, date and author are not within the the document environment, so they will not directly show up in our document. The area before our main document is called preamble. In this specific example we use it to set up the values for the maketitle command for later use in our document. This command will automagically create a titlepage for us. The newpage command speaks for itself.

Obviously the statements title, date and author are not within the the document environment, so they will not directly show up in our document. The area before our main document is called preamble. In this specific example we use it to set up the values for the maketitle command for later use in our document. This command will automagically create a titlepage for us. The newpage command speaks for itself.

Obviously the statements title, date and author are not within the the document environment, so they will not directly show up in our document. The area before our main document is called preamble. In this specific example we use it to set up the values for the maketitle command for later use in our document. This command will automagically create a titlepage for us. The newpage command speaks for itself.

Obviously the statements title, date and author are not within the the document environment, so they will not directly show up in our document. The area before our main document is called preamble. In this specific example we use it to set up the values for the maketitle command for later use in our document. This command will automagically create a titlepage for us. The newpage command speaks for itself.

\begin{center}
	\begin{tabular}{| l | c | r |}
		\hline
		dummy text dummy text  & cell2 & cell3 \\ 
		\hline
		cell4 & cell5 & cell6 \\ 
		\hline 
		cell7 & cell8 & dummy text  \\   
		\hline
	\end{tabular}
\end{center}

\begin{center}
	\begin{tabular}{||c c c c||} 
		\hline
		Col1 & Col2 & Col2 & Col3 \\ 
		\hline
		1 & 6 & 87837 & 787 \\ 
		\hline
		2 & 7 & 78 & 5415 \\
		\hline
		3 & 545 & 778 & 7507 \\
		\hline
		4 & 545 & 18744 & 7560 \\
		\hline
		5 & 88 & 788 & 6344 \\ 
		\hline
	\end{tabular}
\end{center}


\begin{center}
	\begin{tabular}{ | m{6em} | m{2cm}| m{4cm} | } 
		\hline
		cell1 dummy text dummy text dummy text& cell2 & cell3 \\ 
		\hline
		cell1 dummy text dummy text dummy text & cell5 & cell6 \\ 
		\hline
		cell7 & cell8 & cell9 \\ 
		\hline
	\end{tabular}
\end{center}

\begin{tabularx}{0.98\textwidth} { 
		| >{\raggedright\arraybackslash}X 
		| >{\centering\arraybackslash}c 
		| >{\raggedleft\arraybackslash}c | }
	\hline
	cell1 dummy text dummy text dummy text cell1 dummy text dummy text dummy text & item 12 & item 13 \\
	\hline
	item 21  & item 22  & item 23  \\
	\hline
\end{tabularx}

\subsection{Об'єднання комірок}

\begin{tabular}{ |p{3cm}||p{3cm}|p{3cm}|p{3cm}|  }
	\hline
	\multicolumn{4}{|c|}{Country List} \\
	\hline
	Country Name or Area Name& ISO ALPHA 2 Code &ISO ALPHA 3 Code&ISO numeric Code\\
	\hline
	Afghanistan   & AF    &AFG&   004\\
	Aland Islands &   AX  & ALA   &248\\
	Albania 	  &AL & ALB&  008\\
	Algeria       &DZ & DZA&  012\\
	American Samoa&   AS  & ASM&016\\
	Andorra& AD  & AND   &020\\
	Angola& AO  & AGO&024\\
	\hline
\end{tabular}


\begin{center}
	\begin{tabular}{ |c|c|c|c| } 
		\hline
		col1 & col2 & col3 \\
		\hline
		\multirow{3}{4em}{Multiple row} & cell2 & cell3 \\ 
		& cell5 & cell6 \\ 
		& cell8 & cell9 \\ 
		\hline
	\end{tabular}
\end{center}

\section{Довгі таблиці}

\lipsum[1]
\lipsum[1]


%\begin{table}[h] 
%\centering
\newpage
\begin{longtable}{| p{.18\textwidth} | p{.7\textwidth} |} 
	\hline
	foo & bar \\ \hline 
	foo & bar \\ \hline
	foo & bar \\ \hline
	foo & bar \\ \hline
	foo & bar \\ \hline
	foo & bar \\ \hline
	foo & bar \\ \hline
	foo & bar \\ \hline
	foo & bar \\ \hline
	foo & bar \\ \hline
	foo & bar \\ \hline
		foo & bar \\ \hline
	foo & bar \\ \hline
	foo & bar \\ \hline
	foo & bar \\ \hline
	foo & bar \\ \hline
	foo & bar \\ \hline
	foo & bar \\ \hline
		foo & bar \\ \hline
	foo & bar \\ \hline
	foo & bar \\ \hline
	foo & bar \\ \hline
	foo & bar \\ \hline
	foo & bar \\ \hline
	foo & bar \\ \hline
		foo & bar \\ \hline
	foo & bar \\ \hline
	foo & bar \\ \hline
	foo & bar \\ \hline
	foo & bar \\ \hline
	foo & bar \\ \hline
	foo & bar \\ \hline
		foo & bar \\ \hline
	foo & bar \\ \hline
	foo & bar \\ \hline
	foo & bar \\ \hline
	foo & bar \\ \hline
	foo & bar \\ \hline
	foo & bar \\ \hline
		foo & bar \\ \hline
	foo & bar \\ \hline
	foo & bar \\ \hline
	foo & bar \\ \hline
	foo & bar \\ \hline
	foo & bar \\ \hline
	foo & bar \\ \hline
		foo & bar \\ \hline
	foo & bar \\ \hline
	foo & bar \\ \hline
	foo & bar \\ \hline
	foo & bar \\ \hline
	foo & bar \\ \hline
	foo & bar \\ \hline
		foo & bar \\ \hline
	foo & bar \\ \hline
	foo & bar \\ \hline
	foo & bar \\ \hline
	foo & bar \\ \hline
	foo & bar \\ \hline
	foo & bar \\ \hline
		foo & bar \\ \hline
	foo & bar \\ \hline
	foo & bar \\ \hline
	foo & bar \\ \hline
	foo & bar \\ \hline
	foo & bar \\ \hline

	\caption{Your caption here} % needs to go inside longtable environment
	\label{tab:myfirstlongtable}
\end{longtable}
%\end{table} 

Table \ref{tab:myfirstlongtable} shows my first longtable.

\lipsum[1]
