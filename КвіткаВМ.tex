 \documentclass[12pt, a4paper]{article}
\usepackage[utf8]{inputenc}
\usepackage[ukrainian]{babel}
\usepackage{graphicx}
\graphicspath{ {images/} }
\usepackage{float}
\usepackage{multicol}
\usepackage{blindtext}

\title{Практична робота 
	
	Робота в середовищі tinkercad.}
\author{Розробив вчитель інформатики Квітка Володимир Миколайович}
\date{Робота виконується в онлайн-режимі в середовищі tinkercad }

\begin{document}
	
	\maketitle
	\begin{abstract}
		Знайомство з середовищем Tincercad. Створення 3 D обєктів в Tinkercad
	\end{abstract}
	\subsection*{\textit{Цілі:}}
	
	У цьому розділі учні повинні:
	\begin{enumerate}
		\item Отримати базові уявлення про середовище створення 3 D об'єктів Tinkercad.
		\item Ознайомитися з інтерфейсом Tinkercad.
		\item Створити 3 D об'єкт.
	\end{enumerate}
	\begin{center}
		\begin{tabular}{|p{0.025\textwidth} | p{0.75\textwidth}|}
			\hline 
			\multicolumn{2}{|c|}{План уроку}\\
			\hline
			1 & Про tinkercad  \\ 
			\hline
			2 & Створення облікового запису \\ 
			\hline
			3 & Інтерфейс середовища  \\ 
			\hline
			4 & Способи створення дизайнів в tinkercad\\ 
			\hline
		\end{tabular}
	\end{center}
	
	
	
	
	\subsection*{\textit{Ключові моменти:}}
	\begin{enumerate}
		\item Створити обліковий запис.
		\item Вивчити інтерфейс середовища.
		\item Створити 3 D об'єкт на основі базових елементів.
	\end{enumerate}
	
	
	
\end{document}